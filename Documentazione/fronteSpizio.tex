\documentclass[a4paper,12pt,openright,oneside]{book}
\usepackage{graphicx}
\usepackage{amsfonts,amssymb,amstext,amsmath,amsthm,verbatim,times,cancel,epsfig} 
\usepackage[utf8]{inputenc}
\usepackage[italian]{babel}
\usepackage[T1]{fontenc} 
\usepackage{fancyhdr}
\usepackage{setspace}
\usepackage[italian]{varioref}
\usepackage{hyperref}
\hypersetup{
    colorlinks=true, %set true if you want colored links
    linktoc=all,     %set to all if you want both sections and subsections linked
    linkcolor=black,  %choose some color if you want links to stand out
}



\begin{document} 
\begin{titlepage}


\begin{center}
    {\bfseries\Huge Università degli Studi di Napoli\\}
\end{center} 

\begin{figure}[h]
    \begin{center}
        
\includegraphics[width=0.25\textwidth]{fiilogo.png}    %inserire nella stessa cartella del tex il logo dell'università e inserire il nome al posto di logo.jpg
    \end{center}
\end{figure}
  
\begin{center}
    \bf{Scuola Politecnica e delle Scienze di Base}
\end{center}
\begin{center}
    \bf{Area Didattica di Scienze Matematiche Fisiche e Naturali}
\end{center}
\vspace{5pt}
\begin{center}
    \textbf{Dipartimento di Ingegneria Elettrica e delle Tecnologie dell'Informazione}
\end{center}
\vspace{40pt}
\begin{center}
    {\emph{\Large{\bf{Progetto sistemi operativi}}}}\\
    
\end{center}
\vspace{15pt}


\begin{center}
    {{ \textit{Traccia A}}}
\end{center}
\vspace{25mm}
\par
\noindent
\begin{minipage}[t]{0.47\textwidth}
    \textbf{Professore:}\\
    Alberto Finzi\\   
\end{minipage}
\hfill
\begin{minipage}[t]{0.47\textwidth}\raggedleft
    \textbf{Candidati:}\\
    Mario Turco\\ 
    Matr. $N860002503$\\
    Francesco Longobardi\\
    Matr.  $N860002503$\\
\end{minipage}
\vspace{5.5mm}
\begin{center}
    {\large{\bf Anno Accademico 2019/2020}} 
\end{center}
\end{titlepage}
\newpage
\tableofcontents
\newpage
\pagenumbering{arabic}
\fontfamily{lmodern}
\section{Modalità di compilazione}
\paragraph{}
Il progetto è provvisto un file makefile il quale è in grado di compilare autonomamente l'intero progetto.
Per utilizzare il make file aprire la cartella del progetto tramite la console di sistema e digitare "make".
\section{Seconda sezione}
\end{document}