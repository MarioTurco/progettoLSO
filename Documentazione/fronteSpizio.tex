\documentclass[a4paper]{article}
\usepackage{amsfonts,amssymb,amstext,amsmath,amsthm,verbatim,times,cancel,epsfig} 
\usepackage[utf8]{inputenc}
\usepackage[italian]{babel} 
\usepackage{setspace}
\usepackage{hyperref}
\usepackage{fancyvrb}
\usepackage{multicol}
\usepackage{listings}
\usepackage{caption}
\usepackage{xcolor}
\definecolor{RoyalBlue}{cmyk}{1, 0.50, 0, 0}
\hypersetup{
    colorlinks=true, %set true if you want colored links
    linktoc=all,     %set to all if you want both sections and subsections linked
    linkcolor=black,  %choose some color if you want links to stand out
}
\lstset{language=C,
    keywordstyle=\color{RoyalBlue},
    basicstyle=\scriptsize\ttfamily,
    commentstyle=\ttfamily\itshape\color{gray},
    stringstyle=\ttfamily,
    showstringspaces=false,
    breaklines=true,
    frameround=ffff,
    frame=single,
    rulecolor=\color{black},
    numbers = left,
    extendedchars=true,                      %questo rigo ed il successivo servono per usare le lettere à è ù nel codice
    literate={à}{{\'a}}1 {è}{{\'a}}1 {ù}{{\'u}}1 
}
\renewcommand\lstlistlistingname{Listati}
\renewcommand\lstlistingname{Listato}
\def\lstlistingautorefname{List.}
\captionsetup[lstlisting]{font={small,tt}}
\begin{document} 

\begin{titlepage}
  \pagestyle{empty}
  \begin{center}
  
  % TO EDIT ACCORDINGLY TO...
  % -----------------------------------------------------------------------------------------------
  %
  
  % ... University Name
  {\bfseries\Huge Università degli Studi di Napoli\\}
  \vspace{2.54mm}
  {\bfseries\Huge Federico II\\}
  \vspace{5mm}
  
  % ... Logo
  \centerline{\mbox{\includegraphics[width=36mm]{fiilogo.png}}}
  
  % ... Departement
  \medskip
  {\bfseries\LARGE Dipartimento di Ingegneria Elettrica e\\}
  \vspace{2.54mm}
  {\bfseries\LARGE delle Tecnologie dell'Informazione\\}
  \vspace{5mm}
  
 
  
  % ... Degree Course
  {\large Corso di Laurea Triennale in Informatica\\}
  \vspace{2.5mm}
   % ... Degree Class
   {\emph{\large Classe n. L-31\\}}
   \vspace{2.54mm}
  % Change the name of Departement, Class and Cource in English?
  
  %
  % -----------------------------------------------------------------------------------------------
  %
  
  \vfill
  {\emph{\LARGE Progetto di sistemi operativi\\}}		% Thesis title
  \vspace{4mm}
  {\Large Traccia A\\}
  \vspace{4mm}
  
  \vfill
  
  \begin{multicols}{2}
    {\large Professore:\\}
    Finzi Alberto\
    \vspace{5mm}
    
    
      \vspace{5mm}
    
    {\large Candidati:\\}
    Turco Mario\\
    Matr. N8600/2503\\
    Longobardi Francesco\\
    Matr. N8600/2468\\
    \vspace{10mm}
  \end{multicols}
  
  \vfill
  
  % Fill the year
  {\large Anno Accademico\\ 2019/2020}
  
  \end{center}
  

\end{titlepage}
%%%%%%%%%%%%%%%%%%%%%%%%%%%%%%%%%%%%%%%%%%%%%%%%%%%%%%%%%%%%%%%%%%%%%%
\tableofcontents
\clearpage\null\newpage

\pagenumbering{arabic}
\section{Istruzioni preliminari}
\subsection{Modalità di compilazione}
\paragraph{}
Il progetto è provvisto di un file makefile il quale è in grado di compilare autonomamente l'intero progetto.
Per utilizzare il makefile aprire la cartella del progetto tramite la console di sistema e digitare "make".\\
In alternativa è possibile compilare manualemente il client ed il server con i seguenti comandi:
\begin{Verbatim}[fontsize=\small]
    gcc -o server server.c boardUtility.c parser.c list.c -lpthread
    gcc -o client client.c boardUtility.c parser.c list.c -lpthread
\end{Verbatim}
\section{Guida all'uso }
\subsection{Server}
\paragraph{}
Una volta compilato il progetto è possibile avviare il server digitando da console il seguente comando
\begin{verbatim}
    ./server users
\end{verbatim}
L'identificativo \textit{users} ri riferisce al nome del file sul quale sarà salvata la lista degli utenti e delle loro credenziali.\\
È possibile scegliere un nome a piacimento per il file purchè esso sia diverso da \textit{log}.
\subsection{Client}
\paragraph{}
Una volta compilato il progetto è possibile avviare il cient digitando da console il seguente comando:
\begin{verbatim}
    ./client ip porta
\end{verbatim}
Dove \textit{ip} andrà sostituito con l'ip o l'indirizzo URL del server e \textit{porta} andrà sostituito con la porta del server.\\
Una volta avviato il client comprarià il menu con le scelte 3 possibili: accedi, registrati ed esci.\\
Una volta effettuata la registrazione dell'utente è possibile effettuare l'accesso al programma al seguito del quale verranno mostrate sia la mappa dei gioco si le istruzioni di gioco.
\pagebreak
\section{Comunicazione tra client e server}
Di seguito verranno illustrate le modalità di comunicazione tra client e server.
\subsection{Configurazione del server}
Il socket del server viene configurato con famiglia di protoccoli PF\_NET, con tipo di trasmissione dati SOCK\_STREAM e con protocollo TCP. Mostriamo di seguito il codice sorgente:
\lstinputlisting[caption= {Configurazione indirizzo del server}, firstline={132}, lastline={139}]{../server.c}
\lstinputlisting[caption={Configurazione socket del server}, firstline={414}, lastline={428}]{../server.c}
È importante notare anche come il server riesca a gestire in modo concorrente più client tramite l'uso di un thread dedicato ad ogni client. 
Una volta aver configurato il socket, infatti,  il server si mette in ascolto per nuove connessioni in entrata ed ogni volta che
viene stabilita una nuova connessione viene avviato un thread per gestire tale connessione. Di seguito il relativo codice: 
%97 116
\lstinputlisting[caption={Procedura di ascolto del server}, firstline={97}, lastline={131}]{../server.c}

In particolare al rigo 31 notiamo la creazione di un nuovo thread per gestire la connessione in entrata a cui passiamo il descrittore del client di cui si deve occupare.
\pagebreak
\subsection{Configurazione del client}
Il cliente invece viene configurato e si connette al server tramite la seguente funzione:
\lstinputlisting[caption={Configurazione e connessione del client}, firstline={65}, lastline={83}]{../client.c}

Si noti come al rigo 9 viene configurato il socket ed al rigo 13 viene invece effettuato il tentativo di connessione al server.\\
Al rigo 3 invece viene convertita la porta inserita in input (argv$[$2$]$) dal tipo stringa al tipo della porta (uint16\_t ovvero unsigned long integer).\\
Al rigo 4 notiamo invece la risoluzione dell'url da parte della funzione ipResolver che è riportata di seguito:
\lstinputlisting[caption={Risoluzione url del client}, firstline={250}, lastline={261}]{../client.c}
Al rigo 4, tramite l'url o l'indirizzo ip viene riempita la struttura hostent da cui poi possiamo estrarre l'indirizzo ip presente nel campo 
h\_addr\_list  che, in effetti, è un array che contiene i vari indirizzi ip associati a quell'host.\\
Infine, al rigo 11 decidiamo di ritoranre soltanto il primo indirizzo convertito in Internet dot notation.
\pagebreak
\subsection{Comunicazione tra client e server}
La comunicazione tra client e server avviene tramite write e read sul socket.\\
Il comportamento del server e del client è determinato da particolari messaggi inviati e/o ricevuti che codificano, tramite interi o caratteri, la richiesta da parte del client di usufruire di un determinato servizio e la relativa risposta del server.
 \subsubsection{Esempio: la prima comunicazione}
 In particolare, una volta effettuata la connessione,
il server attenderà un messaggio dal client per poter avviare una delle tre possibili procedure, ovvero login, registrazione ed uscita ( rispettivamente codici: 1,2,3).\\
Di seguito sono riportate le realtive funzioni di gestione che entrano in esecuzione subito dopo aver stabilito la connessione
tra client e server.
\lstinputlisting[caption={Prima comunicazione del server}, label={alg:serverPreLogin}, firstline={180}, lastline={201}]{../server.c}
Si noti come il server riceva, al rigo 7, il messaggio codificato da parte del client e metta in esecuzione la funzione corrispondente.
\lstinputlisting[caption={Prima comunicazione del client}, firstline={84}, lastline={102}, label={alg:clientPreLogin} ]{../client.c}

\section{Comunicazione durante la partita}
Una volta effettuato il login, il client potrà iniziare a giocare tramite la funzione play (Vedi \autoref{alg:playServer} e \autoref{alg:playclient})
che rappresentano il cuore della comunicazione tra client e server.
\subsection{Funzione core del server}
La funzione play del server consiste di un ciclo infinito nel quale il server invia al client tre informazioni importanti:
\begin{itemize}
  \setlength\itemsep{0.2mm}
  \item La griglia di gioco (Rigo 26)
  \item Il player con le relative informazioni (Rigo 28 a 31)
  \item Il timer (Rigo 32 )
  \item Un messaggio che notifica al client se è iniziato un nuovo turno oppure no
\end{itemize}
Dopodichè il thread del server rimane in attesa di ricevere l'input del client per spostare il giocaore sulla mappa tramite la relativa funzione. (Rigo 34)
Oltre questo, la funzione play del server si occupa anche di generare la posizione del player appena entra in partita e generare la nuova posizione del player (Rigo 52 a 66) quando viene effettuato il cambio di mappa (allo scadere del tempo o alla raggiunta del massimo punteggio).

\subsection{Funzione core del client}
Dall'altro lato, la funzione play del client, è stata mantenuta volutamente semplice.
Lo scopo del client è unicamente quello di ricevere i dati forniti dal server, stampare la mappa di gioco e leggere l'input da inviare al server.
Unica eccezzione è il rigo 30  del client che non richiede la ricezione di ulteriori dati dal server: al rigo 30, infatti si avvia la procedura di disconnessione del client .
\pagebreak
\lstinputlisting[caption={Funzione play del server}, label={alg:playServer}, firstline={202}, lastline={269}]{../server.c}

\pagebreak
\lstinputlisting[caption={Funzione play del client}, label={alg:playclient}, firstline={111}, lastline={149}]{../client.c}
\appendix
\section{Codici sorgente}
Di seguito sono riportati tutti i codici sorgenti integrali del progetto.
\subsection{Codice sorgente del client}
\lstinputlisting[caption ={ Codice sorgente del client}]{../client.c}
\subsection{Codice sorgente del server}
\lstinputlisting[caption ={Codice sorgente del server} ]{../server.c}
\subsection{Codice sorgente boardUtility}
\lstinputlisting[caption ={Codice header utility del gioco 1} ]{../boardUtility.h}
\lstinputlisting[caption ={Codice sorgente utility del gioco 1} ]{../boardUtility.c}
\subsection{Codice sorgente list}
\lstinputlisting[caption ={Codice header utility del gioco 2} ]{../list.h}
\lstinputlisting[caption ={Codice sorgente utility del gioco 2} ]{../list.c}
\subsection{Codice sorgente parser}
\lstinputlisting[caption ={Codice header utility del gioco 3} ]{../parser.h}
\lstinputlisting[caption ={Codice sorgente utility del gioco 3} ]{../parser.c}
\pagebreak
\lstlistoflistings


\end{document}