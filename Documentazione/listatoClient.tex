\begin{lstlisting}
    #include "boardUtility.h"
  #include "list.h"
  #include "parser.h"
  #include <arpa/inet.h>
  #include <fcntl.h>
  #include <netdb.h>
  #include <netinet/in.h> //conversioni
  #include <netinet/in.h>
  #include <netinet/ip.h> //struttura
  #include <pthread.h>
  #include <signal.h>
  #include <stdio.h>
  #include <stdlib.h>
  #include <string.h>
  #include <sys/socket.h>
  #include <sys/stat.h>
  #include <sys/types.h>
  #include <time.h>
  #include <unistd.h>
  
  void printPlayerList();
  int getTimer();
  void printTimer();
  void play();
  int tryLogin();
  void printMenu();
  int connettiAlServer(char **argv);
  char *ipResolver(char **argv);
  int registrati();
  int gestisci();
  char getUserInput();
  void clientCrashHandler();
  void serverCrashHandler();
  int serverCaduto();
  void esciDalServer();
  int isCorrect(char);
  
  int socketDesc;
  char grigliaDiGioco[ROWS][COLUMNS];
  
  int main(int argc, char **argv) {
    signal(SIGINT, clientCrashHandler); /* CTRL-C */
    signal(SIGHUP, clientCrashHandler); /* Chiusura della console */
    signal(SIGQUIT, clientCrashHandler);
    signal(SIGTSTP, clientCrashHandler); /* CTRL-Z*/
    signal(SIGTERM, clientCrashHandler); /* generato da 'kill' */
    signal(SIGPIPE, serverCrashHandler);
    char bufferReceive[2];
    if (argc != 3) {
      perror("Inserire indirizzo ip/url e porta (./client 127.0.0.1 5200)");
      exit(-1);
    }
    if ((socketDesc = connettiAlServer(argv)) < 0)
      exit(-1);
    gestisci(socketDesc);
    close(socketDesc);
    exit(0);
  }
  void esciDalServer() {
    int msg = 3;
    printf("Uscita in corso\n");
    write(socketDesc, &msg, sizeof(int));
    close(socketDesc);
  }
  int connettiAlServer(char **argv) {
    char *indirizzoServer;
    uint16_t porta = strtoul(argv[2], NULL, 10);
    indirizzoServer = ipResolver(argv);
    struct sockaddr_in mio_indirizzo;
    mio_indirizzo.sin_family = AF_INET;
    mio_indirizzo.sin_port = htons(porta);
    inet_aton(indirizzoServer, &mio_indirizzo.sin_addr);
    if ((socketDesc = socket(PF_INET, SOCK_STREAM, 0)) < 0)
      perror("Impossibile creare socket"), exit(-1);
    else
      printf("Socket creato\n");
    if (connect(socketDesc, (struct sockaddr *)&mio_indirizzo,
                sizeof(mio_indirizzo)) < 0)
      perror("Impossibile connettersi"), exit(-1);
    else
      printf("Connesso a %s\n", indirizzoServer);
    return socketDesc;
  }
  int gestisci() {
    char choice;
    while (1) {
      printMenu();
      scanf("%c", &choice);
      fflush(stdin);
      system("clear");
      if (choice == '3') {
        esciDalServer();
        return (0);
      } else if (choice == '2') {
        registrati();
      } else if (choice == '1') {
        if (tryLogin())
          play();
      } else
        printf("Input errato, inserire 1,2 o 3\n");
    }
  }
  int serverCaduto() {
    char msg = 'y';
    if (read(socketDesc, &msg, sizeof(char)) == 0)
      return 1;
    else
      write(socketDesc, &msg, sizeof(msg));
    return 0;
  }
  void play() {
    PlayerStats giocatore = NULL;
    int score, deploy[2], position[2], timer;
    int turnoFinito = 0;
    int exitFlag = 0, hasApack = 0;
    while (!exitFlag) {
      if (serverCaduto())
        serverCrashHandler();
      if (read(socketDesc, grigliaDiGioco, sizeof(grigliaDiGioco)) < 1)
        printf("Impossibile comunicare con il server\n"), exit(-1);
      if (read(socketDesc, deploy, sizeof(deploy)) < 1)
        printf("Impossibile comunicare con il server\n"), exit(-1);
      if (read(socketDesc, position, sizeof(position)) < 1)
        printf("Impossibile comunicare con il server\n"), exit(-1);
      if (read(socketDesc, &score, sizeof(score)) < 1)
        printf("Impossibile comunicare con il server\n"), exit(-1);
      if (read(socketDesc, &hasApack, sizeof(hasApack)) < 1)
        printf("Impossibile comunicare con il server\n"), exit(-1);
      timer = getTimer();
      giocatore = initStats(deploy, score, position, hasApack);
      printGrid(grigliaDiGioco, giocatore);
      char send = getUserInput();
      write(socketDesc, &send, sizeof(char));
      read(socketDesc, &turnoFinito, sizeof(turnoFinito));
      if (turnoFinito) {
        system("clear");
        printf("Turno finito\n");
        sleep(1);
      } else {
        if (send == 'e' || send == 'E')
          printf("Disconnessione in corso...\n"), exit(0);
        if (send == 't' || send == 'T')
          printTimer();
        else if (send == 'l' || send == 'L')
          printPlayerList();
      }
    }
  }
  void printPlayerList() {
    system("clear");
    int lunghezza = 0;
    char buffer[100];
    int continua = 1;
    int number = 1;
    fprintf(stdout, "Lista dei player: \n");
    if (!serverCaduto(socketDesc)) {
      read(socketDesc, &continua, sizeof(continua));
      while (continua) {
        read(socketDesc, &lunghezza, sizeof(lunghezza));
        read(socketDesc, buffer, lunghezza);
        buffer[lunghezza] = '\0';
        fprintf(stdout, "%d) %s\n", number, buffer);
        continua--;
        number++;
      }
      sleep(1);
    }
  }
  void printTimer() {
    int timer;
    if (!serverCaduto(socketDesc)) {
      read(socketDesc, &timer, sizeof(timer));
      printf("\t\tTempo restante: %d...\n", timer);
      sleep(1);
    }
  }
  int getTimer() {
    int timer;
    if (!serverCaduto(socketDesc))
      read(socketDesc, &timer, sizeof(timer));
    return timer;
  }
  int tryLogin() {
    int msg = 1;
    write(socketDesc, &msg, sizeof(int));
    system("clear");
    printf("Inserisci i dati per il Login\n");
    char username[20];
    char password[20];
    printf("Inserisci nome utente(MAX 20 caratteri): ");
    scanf("%s", username);
    printf("\nInserisci password(MAX 20 caratteri):");
    scanf("%s", password);
    int dimUname = strlen(username), dimPwd = strlen(password);
    if (write(socketDesc, &dimUname, sizeof(dimUname)) < 0)
      return 0;
    if (write(socketDesc, &dimPwd, sizeof(dimPwd)) < 0)
      return 0;
    if (write(socketDesc, username, dimUname) < 0)
      return 0;
    if (write(socketDesc, password, dimPwd) < 0)
      return 0;
    char validate;
    int ret;
    read(socketDesc, &validate, 1);
    if (validate == 'y') {
      ret = 1;
      printf("Accesso effettuato\n");
    } else if (validate == 'n') {
      printf("Credenziali Errate o Login gia' effettuato\n");
      ret = 0;
    }
    sleep(1);
    return ret;
  }
  int registrati() {
    int msg = 2;
    write(socketDesc, &msg, sizeof(int));
    char username[20];
    char password[20];
    system("clear");
    printf("Inserisci nome utente(MAX 20 caratteri): ");
    scanf("%s", username);
    printf("\nInserisci password(MAX 20 caratteri):");
    scanf("%s", password);
    int dimUname = strlen(username), dimPwd = strlen(password);
    if (write(socketDesc, &dimUname, sizeof(dimUname)) < 0)
      return 0;
    if (write(socketDesc, &dimPwd, sizeof(dimPwd)) < 0)
      return 0;
    if (write(socketDesc, username, dimUname) < 0)
      return 0;
    if (write(socketDesc, password, dimPwd) < 0)
      return 0;
    char validate;
    int ret;
    read(socketDesc, &validate, sizeof(char));
    if (validate == 'y') {
      ret = 1;
      printf("Registrato con successo\n");
    }
    if (validate == 'n') {
      ret = 0;
      printf("Registrazione fallita\n");
    }
    sleep(1);
    return ret;
  }
  char *ipResolver(char **argv) {
    char *ipAddress;
    struct hostent *hp;
    hp = gethostbyname(argv[1]);
    if (!hp) {
      perror("Impossibile risolvere l'indirizzo ip\n");
      sleep(1);
      exit(-1);
    }
    printf("Address:\t%s\n", inet_ntoa(*(struct in_addr *)hp->h_addr_list[0]));
    return inet_ntoa(*(struct in_addr *)hp->h_addr_list[0]);
  }
  void clientCrashHandler() {
    int msg = 3;
    int rec = 0;
    printf("\nChiusura client...\n");
    do {
      write(socketDesc, &msg, sizeof(int));
      read(socketDesc, &rec, sizeof(int));
    } while (rec == 0);
    close(socketDesc);
    signal(SIGINT, SIG_IGN);
    signal(SIGQUIT, SIG_IGN);
    signal(SIGTERM, SIG_IGN);
    signal(SIGTSTP, SIG_IGN);
    exit(0);
  }
  void serverCrashHandler() {
    system("clear");
    printf("Il server e' stato spento o e' irraggiungibile\n");
    close(socketDesc);
    signal(SIGPIPE, SIG_IGN);
    premiEnterPerContinuare();
    exit(0);
  }
  char getUserInput() {
    fflush(stdin);
    char c;
    c = getchar();
    int daIgnorare;
    while ((daIgnorare = getchar()) != '\n' && daIgnorare != EOF) {
    }
    return c;
  }
  \end{lstlisting}